%!TEX root=ClassNotes.tex

\section{Taylor Series}
Taylor series is a technique used to approximate functions using {\it polynomials}.

\begin{definition}
  \label{def:Taylor_series}
For a positive integer $n$, a real number $a$, and an infinitely differentiable function $f$, the {\bf n$^{th}$ Taylor polynomial/approximation centered at $a$}, denoted $T_nf (x)$, is a degree $n$ polynomial defined as\\
  \begin{align*}
    T_nf (x) &= f(a)
    + f'(a) \cdot \dfrac{(x-a)}{1}
    + f''(a) \cdot \dfrac{(x-a)^2}{2!}
    + \dots
    + f^{(n)}(a) \cdot \dfrac{(x-a)^n}{n!}\\
  \end{align*}
  where $f^{(n)}(a)$ denotes the $n^{th}$ derivative of $f$ at $a$.\footnote{Recall that $n! = n \cdot (n-1) \cdot (n-2) \cdots 2 \cdot 1 $.}
The limit $n \rightarrow \infty$ of the above series is called the {\bf Taylor series $Tf(x)$}.\\
\begin{align*}
  Tf(x) &= f(a)
  + f'(a) \cdot \dfrac{(x-a)}{1}
  + f''(a) \cdot \dfrac{(x-a)^2}{2!}
  + f^{(3)}(a) \cdot \dfrac{(x-a)^3}{3!}
  + \dots\\
\end{align*}
\end{definition}

\begin{remark}
  The Taylor series defined above is just a {\it formal series}.
  It is not always possible to plug in a value for $x$ because of convergence issues.
  We'll discuss this in the later sections.
\end{remark}

We'll mostly be interested in the Taylor series centered at $a = 0$, \\
\begin{align*}
  Tf(x)
  &=
  f(0)
  + f'(0) \cdot \dfrac{x}{1}
  + f''(0) \cdot \dfrac{x^2}{2!}
  + f^{(3)}(0) \cdot \dfrac{x^3}{3!}
  + \dots\\
\end{align*}
In this case, $Tf(x)$ is called the $n^{th}$ {\bf Maclaurin series}.



\begin{exercise}
  For each of the following functions compute $f^{(n)}(0)$, for positive integers $n$, and use these to compute the Maclaurin series.
  \begin{multicols}{2}
  \begin{enumerate}
    \item $e^x$
    \item $\sin x$
    \item $\cos x$
    \item $\ln {(1+x)}$
    \item $\dfrac{1}{1-x}$
    \item (Optional) $\tan^{-1} x$
  \end{enumerate}
  \end{multicols}
\end{exercise}

\begin{exercise}
  This problem explains why the Definition \ref{def:Taylor_series} is the ``correct'' definition for Taylor series.
  \begin{enumerate}
    \item Let $f(x) = x^k$. Compute $T_n f(x)$ centered at $0$ for
    \begin{enumerate}
      \item $n < k$
      \item $n \ge k$
    \end{enumerate}
    \item More generally, let
    \begin{align*}
      f(x) = a_0 + a_1 x + a_2 x^2 + \dots + a_k x^k
    \end{align*}
    Using the previous part, compute $T_n f(x)$ for
    \begin{enumerate}
      \item $n < k$
      \item $n \ge k$
    \end{enumerate}
     What is the Maclaurin series for this $f(x)$?
  \end{enumerate}
\end{exercise}
The above statement is more generally for all Taylor series centered at any point $a$.
\begin{exercise}{\bf (Optional)}
  Show that the Taylor series of $f(x) = x^n$, centered at a real number $a$, equals $x^n$. Argue that this is more generally true for an arbitrary polynomial $f(x)$.
\end{exercise}

\subsection{Remainder Term}
As mentioned earlier, Definition \ref{def:Taylor_series} defines a \textit{formal series} and it not possible to make sense of $Tf(x)$ for a real number $x$.
The two important questions that we need to answer are:
\begin{enumerate}
  \item  Does the limit $\lim \limits_{n \rightarrow \infty}T_n f(x)$ exist?
  \item Does the limit $\lim \limits_{n \rightarrow \infty}T_n f(x)$ equal $f(x)$?
\end{enumerate}
Unless the answer to both the questions is {\it yes} it is not possible to use Taylor polynomials for approximating the function $f(x)$.

Both the questions are in generally difficult to answer.
To tackle the first question we need techniques from series and sequences.
In this section, we'll focus on answering the second question which can be done using basic Calculus.


\begin{definition}
  For a smooth function $f$ and real numbers $x$, $a$, the {\bf $n^{th}$ error term} or the {\bf $n^{th}$ remainder term $R_nf(x)$} is defined as
  \begin{align*}
    R_nf(x)
    &= f(x) - T_nf(x)
  \end{align*}
  where $T_n f(x)$ is the $n^{th}$ Taylor approximation of $f$ centered at $a$, so that $f(x) = T_nf(x) + R_nf(x)$.
Thus we can say that
\begin{align*}
  \lim \limits_{n \rightarrow \infty}T_n f(x) = f(x)
\end{align*}
if and only if
\begin{align*}
  \lim \limits_{n \rightarrow \infty}R_n f(x) = 0.
\end{align*}
\end{definition}

\newpage

The following Theorem allows us to compute this remainder term using integrals.
\begin{theorem}
  \label{thm:remainder_term}
  With the notation as above, the remainder term is given by
  \begin{align*}
    R_nf(x) = \dfrac{1}{n!} \cdot \int_a^x {(x-t)^n \cdot {f^{(n+1)}(t)} } \: dt
  \end{align*}
\end{theorem}

\begin{exercise}
  \begin{enumerate}
    \item Using integration by parts (if necessary), compute
    \begin{enumerate}
      \item $\int_0^x {f'(t)} \: dt$
      \item $\int_0^x (x-t) \cdot {f''(t)} \: dt$
      \item $\int_0^x (x-t)^2 \cdot {f^{(3)}(t)} \: dt$
    \end{enumerate}
    \item{\bf (Optional)} By repeatedly applying integration by parts, prove that
    \begin{align*}
      \dfrac{1}{n!} \cdot \int_0^x {(x-t)^n \cdot {f^{(n+1)}(t)}} \: dt
    \end{align*}
    equals
    \begin{align*}
      f(x) - \left( f(0)
      + f'(0) \cdot \dfrac{x}{1}
      + f''(0) \cdot \dfrac{x^2}{2!}
      + \dots
      + f^{(n)}(0) \cdot \dfrac{x^n}{n!}
      \right)
    \end{align*}
    thereby proving Theorem \ref{thm:remainder_term} for $a=0$.
  \end{enumerate}
\end{exercise}

\begin{exercise}{\bf (Optional)}
  Prove Theorem \ref{thm:remainder_term} for arbitrary real number $a$.
\end{exercise}


\newpage
\subsection{Computing using Taylor Series}
We can use the Taylor series (centered at 0) to do computations if
\begin{enumerate}
  \item we can compute all the derivatives $f^{(n)}(0)$ for all positive integers $n$
  \item $ \lim \limits_{n \rightarrow \infty}R_n f(x) = 0$.
\end{enumerate}
The condition $ \lim \limits_{n \rightarrow \infty}R_n f(x) = 0$ is highly non-trivial and not checking it leads to absurd results.

\begin{exercise}
  Let $f(x) = \dfrac{1}{1-x}$.
  What happens to the value of the Maclaurin series $Tf(x)$ for $x=2$? What is $f(2)$?
\end{exercise}
For $f(x) = \dfrac{1}{1-x}$ we can show that $ \lim \limits_{n \rightarrow \infty}R_n f(2) = \infty$ (and not 0) and hence the difference between the Taylor series $Tf(2)$ and the function $f(2)$ blows up to infinity.
In the next section, we'll prove the following theorem which says that this does not happen for exponential and trigonometric functions.
\begin{theorem}
  \label{thm:Taylor_series_exponential}
  If $f(x) = e^x$, $\sin x$, and $\cos x$, then $\lim \limits_{n \rightarrow \infty}R_n f(x) = 0$ for \textit{all real numbers} x.
\end{theorem}

Hence, for all real numbers $x$, $\lim \limits_{n \rightarrow \infty}T_n f(x) = f(x)$ for exponential and trigonometric functions and we can use the Taylor series to approximate.

\begin{exercise}
  Use the Maclaurin series of $e^x$ to compute the value of $e$ correct up to 2 decimal places.
\end{exercise}

\begin{exercise}
  \begin{enumerate}
    \item Theorem \ref{thm:Taylor_series_exponential} is also true for $\tan^{-1} x$, the proof is easy but technical. The Maclaurin series of $\tan^{-1} x$ is given by
    \begin{align*}
      x - \dfrac{x^3}{3} + \dfrac{x^5}{5} - \dfrac{x^7}{7} + \dfrac{x^9}{9} + - \dots
    \end{align*}
    Using this find an expression (as an infinite sum) for $\pi$.\hint{Use $x = 1$.}
    \item Use the first 10 terms of this sum to find an approximate value for $\pi$.
    This method for computing $\pi$ is not used in practice as the \textit{rate of convergence} of the Maclaurin series is very slow.
  \end{enumerate}
\end{exercise}

\begin{exercise}
  Theorem \ref{thm:Taylor_series_exponential} is true even if $x$ is a complex number (the proof requires complex analysis).
  Compute the Taylor series of $e^{i\theta}$, $\cos \theta$, $\sin \theta$. Use these to prove the {\it Euler's identity},
\begin{align*}
    e^{i\theta} &= \cos \theta + i \sin \theta
\end{align*}
\end{exercise}



\subsection{Estimating the Remainder Term}
Finally, we  want to prove Theorem \ref{thm:Taylor_series_exponential}.

\begin{exercise} {\bf (Optional)}
  Let $x$ be a positive real number.
   For each of the following functions $f(x)$, show that there is a constant $M$ such that for all $ t \in (0,x)$  and all positive integers $n$ we have $|f^{(n)}(t)| < M$. (Note that $M$ can depend on $x$ but not on $n$ or $t$.)
   \begin{multicols}{2}
     \begin{enumerate}
       \item $e^x$
       \item $\sin x$
       \item $\cos x$
     \end{enumerate}
   \end{multicols}
\end{exercise}
Using the above problem, the proof of Theorem \ref{thm:Taylor_series_exponential} will be complete once we prove the following Proposition.
\begin{prop}
  Let $x$ be a positive real number. If there is a real number $M$ such that for all $ t \in (0,x)$  and all positive integers $n$ we have $|f^{(n)}(x)| < M$ then $R_n f(x) \rightarrow 0$ as $n \rightarrow \infty$.
\end{prop}
\begin{proof}
  By definition,
  \begin{align*}
    R_n f(x) = \dfrac{1}{n!} \cdot \int_0^x {(x-t)^n \cdot {f^{(n+1)}(t)}} \: dt
  \end{align*}
  Because $(x-t)^n < x^n$ and $|f^{(n+1)}(t)| < M$ for all $0 < t < x$  and all positive integers $n$, we get
  \begin{align*}
    |R_n f(x)|
    &=
    \left|\dfrac{1}{n!} \cdot \int_0^x {(x-t)^n \cdot {f^{(n+1)}(t)}} \: dt \right|\\
    &< \left|\dfrac{1}{n!} \cdot \int_0^x x^n \cdot M \: dt \right| \\
    &=  \left|\dfrac{1}{n!} \cdot x^n \cdot M \int_0^x 1 \: dt \right| \\
    &=  \left|\dfrac{1}{n!} \cdot x^n \cdot M x \right| \\
    &= \dfrac{M x^{n+1}}{n!}
  \end{align*}
  One can show that for any real number $x$, $\lim \limits_{n \rightarrow \infty} x^{n+1} / n! = 0$ (this is easy, see if you can work out the details) so that
\begin{align*}
    \lim \limits_{n \rightarrow \infty} |R_n f(x)|
    &<
    \lim \limits_{n \rightarrow \infty} \dfrac{M x^{n+1}}{n!}
    \\
    &= {M}\cdot \lim \limits_{n \rightarrow \infty} \dfrac{x^{n+1}}{n!} \\
    &= M \cdot 0 = 0
\end{align*}
\end{proof}

To summarize, Taylor series provides us a tool for approximating functions using polynomials (in the cases where the remainder term tends to 0).
This is a very common technique in analysis: we try to approximate a function by a series of simpler functions and show that the difference between the two tends to 0.
Fourier series, for example, does this using trigonometric functions.
