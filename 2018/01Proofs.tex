%!TEX root=ClassNotes.tex

\section{Proofs}

We'll start by learning about the various kinds of proofs that we'll encounter in this class.\footnote{This section in taken almost verbatim from \url{http://zimmer.csufresno.edu/~larryc/proofs/proofs.html}.}

Proofs are the heart of mathematics. You must come to terms with proofs -- you must be able to read, understand
and write them. What is the secret? What magic do you need to know? The short answer is: there is no secret, no
mystery, no magic. All that is needed is some common sense and a basic understanding of a few trusted and easy
to understand techniques.

\subsection*{The Structure of a Proof}
The basic structure of a proof is easy: it is just a series of statements, each one being either
\begin{itemize}
	\item
	An assumption or
	\item
	A conclusion, clearly following from an assumption or previously proved result
\end{itemize}
And that is all. Occasionally there will be the clarifying remark, but this is just for the reader and has no logical
bearing on the structure of the proof.

A well written proof will flow. That is, the reader should feel as though they are being taken on a ride that takes
them directly and inevitably to the desired conclusion without any distractions about irrelevant details.

Each step
should be clear or at least clearly justified. A good proof is easy to follow.
When you are finished with a proof, apply the above simple test to every sentence: is it clearly
\begin{enumerate}
	\item an assumption
	\item a justified conclusion?
\end{enumerate}
If the sentence fails the test, maybe it doesn't belong in the proof.


\subsection*{Example}
In order to write proofs, you must be able to read proofs. See if you can follow the proof below. Don't worry
about how you would have (or would not have) come up with the idea for the proof. Read the proof with an eye
towards the criteria listed above. Is each sentence clearly an assumption or a conclusion? Does the proof flow?
Was the theorem in fact proved?

\begin{theorem}
	\label{theorem:irrationality_of_sqrt2}
	 The square root of 2 is an irrational number.
\end{theorem}
\begin{proof}
	Let's represent the square root of 2 by $s$. Then, by definition, $s$ satisfies the equation
	\begin{align*}
		s^2 = 2.
	\end{align*}
	If $s$ were a rational number, then we could write $s = p/q$
where $p$ and $q$ are a pair of integers. In fact, by dividing out the common multiple if necessary, we may even assume $p$ and $q$ have no common multiple (other than 1). If we now substitute this into the first equation we obtain, after a little algebra, the equation
\begin{align*}
		p^2 = 2 q^2.
\end{align*}

But now, 2 must appear in the prime factorization of the number $p^2$ (since it appears in the same number $2 q^2$). Since 2 itself is a prime number, 2 must then appear in the prime factorization of the number $p$. But then, $2 \cdot 2$ would appear in the prime factorization of $p^2$, and hence in $2 q^2$. By dividing out a 2, it then appears that 2 is in the prime factorization of $q^2$. Like before (with $p^2$) we can now conclude 2 is a prime factor of $q$. But now we have $p$ and $q$ sharing a prime factor, namely 2. This violates our assumption above (see if you can find it) that $p$ and $q$ have no common multiple other than 1.
\end{proof}

\subsection{Direct Proofs}
Most theorems that you want to prove are either explicitly or implicitly in the form \begin{align*}
	\mbox{ ``If \textit{P}, then \textit{Q}". }
\end{align*} This is the standard form of a
theorem (though it can be disguised). A direct proof should be thought of as a flow of implications beginning
with \textit{P} and ending with \textit{Q}.
	\begin{align*}
		{P} \implies \dots \implies {Q}
	\end{align*}
Most proofs are (and should be) direct proofs. Always try direct proof first, unless you have a good reason not to. If you find a simple proof, and you are convinced of its correctness, then don't be shy about. Many times proofs
are simple and short.

\begin{exercise}
	\label{question:ex1}
	Prove each of the following.
	\begin{enumerate}
		\item If $a$ divides $b$ and $a$ divides $c$ then $a$ divides $b + c$, where $a$, $b$, and $c$ are positive integers.
		\item For all real numbers $a$ and $b$, $a^2 + b^2 \ge 2 a b$.
		\item If $a$ is a rational number and $b$ is a rational number, then $a+b$ is a rational number.
	\end{enumerate}
\end{exercise}



\subsection{Proof by Contradiction}
\label{section:proofByContradiction}
In a proof by contradiction we assume, along with the hypotheses, the logical negation of the result we wish to
prove, and then reach some kind of contradiction. That is, if we want to prove ``If \textit{P}, then \textit{Q}'', we assume \textit{P} and
Not \textit{Q}. The contradiction we arrive at could be some conclusion contradicting one of our assumptions, or
something obviously untrue like 1 = 0. The proof of Theorem \ref{theorem:irrationality_of_sqrt2} is an example of this.

\begin{exercise}
	\label{question:ex2}
	Use the method of Proof by Contradiction to prove each of the following.
	\begin{enumerate}
		\item The cube root of 2 is irrational.
		\item If $a$ is a rational number and $b$ is an irrational number, then $a+b$ is an irrational number.
		\item There are infinitely many prime numbers.\footnote{There are dozens of proofs of this theorem, originally due to Euclid. Feel free to look one up online.}
	\end{enumerate}
\end{exercise}




\subsection{Proof by Contrapositive}
\label{section:proofByContrapositive}
	Proof by contrapositive takes advantage of the logical equivalence between ``\textit{P} implies \textit{Q}'' and ``Not \textit{Q} implies Not \textit{P}''. For example, the assertion ``If it is my car, then it is red'' is equivalent to ``If that car is not red, then it is not mine''. So, to prove ``If \textit{P}, then \textit{Q}'' by the method of contrapositive means to prove
	\begin{align*}
		\mbox{``If Not \textit{Q}, then Not \textit{P}''.}
	\end{align*}

\noindent \textbf{How Is This Different From Proof by Contradiction?}
The difference between the Contrapositive method and the Contradiction method is subtle. Let's examine how the two methods work when trying to prove ``If \textit{P}, then \textit{Q}".
\begin{description}
	\item[Method of Contradiction:] Assume \textit{P} and Not \textit{Q} and prove some sort of contradiction.
	\item[Method of Contrapositive:] Assume Not \textit{Q} and prove Not \textit{P}.
\end{description}
The method of Contrapositive has the advantage that your goal is clear: Prove Not P. In the method of Contradiction, your goal is to prove a contradiction, but it is not always clear what the contradiction is going to be at the start.

\begin{exercise}
	\label{question:ex3}
	Use the method of Proof by Contrapositive to prove each of the following.
	\begin{enumerate}
		\item If the product of two integers is even, then at least one of the two must be even.
		\item If the product of two integers is odd, then both must be odd.
		\item If the product of two real numbers is an irrational number, then at least one of the two must be an irrational number.
	\end{enumerate}
\end{exercise}


\subsection{Converse}
The converse of an assertion in the form ``If \textit{P}, then \textit{Q}" is the assertion
\begin{align*}
	\mbox{``If \textit{Q}, then \textit{P}"}.
\end{align*}
A common logical fallacy is to assume that if an assertion is true then so is its converse.


\subsubsection{If and Only If}
Many theorems are stated in the form ``\textit{P}, if, and only if, \textit{Q}". Another way to say the same thing is: ``\textit{Q} is necessary, and sufficient for \textit{P}". This means two things:
\begin{center}
	``If \textit{P}, Then \textit{Q}" \qquad and \qquad ``If \textit{Q}, Then \textit{P}".
\end{center} So to prove an ``If, and Only If" theorem, you must prove the theorem and also its \textit{converse}.

\begin{exercise}
	Go back to the problems in Exercises \ref{question:ex1}, \ref{question:ex2}, \ref{question:ex3} and find the ones which are of the form ``If \textit{P}, Then \textit{Q}". For each of these:
	\begin{itemize}
		\item State the converse.
		\item Prove or disprove the converse (by providing either a proof or a counterexample).
		\item For the problems where the converse is also true rewrite the assertion as an "If, and Only If" statement.
	\end{itemize}
\end{exercise}








\subsection{Quantifiers}
It is extremely important in mathematics to be able to formulate very precise statements. To prevent ambiguous statements and logical fallacies the vocabulary used is very limited and every term has a well-defined meaning. The two concepts we need to get used to are {\it logical operators} and {\it quantifiers}.\\

{\bf Logical Operators:} Operators allow us to form complex statements by combining simpler ones. The basic logical operators are {\it and}, {\it or}, and {\it not}. Their usage is the same as in everyday language. The more complicated logical operator that we'll be using a lot is {\it If - then -}. We've already encountered several examples of its usage in the previous sections.\\


{\bf Quantifiers:} Quantifiers allow us to make abstract statements which are ``universally true'' without having to specify a concrete element. There are two commonly used	quantifiers:
\begin{align*}
	\mbox{\textbf{For all/every}} && \mbox{\textbf{There exists}}
\end{align*}
Understanding and formulating complex statements using these requires a lot of practice. You'll get used to these as the course progresses. Here are a few examples,
\begin{example}$ $
	\begin{enumerate}
		\item \emph{For every} odd integer $a$, the integer $a+1$ is even.
		\item An integer $a$ is even if and only if \emph{there exists} an integer $b$ such that $a = 2b$.
		\item There \emph{do not exist} integers $p,q$ such that $\frac{p}{q} = \sqrt{2}$.
		\item \emph{For every} non-zero rational number $x$ \emph{there exists} a rational number $y$ such that $x \cdot y = 1$.
	\end{enumerate}
\end{example}

\subsubsection{Nesting Quantifiers}
	When dealing with statements involving multiple quantifiers the order really matters; changing the order changes the meaning of a statement completely. When you're using multiple quantifiers in a single sentence you should always pause and check if you're using the right quantifiers in the right order.

\begin{exercise}
	For each of the following pairs, explain how changing the order of the quantifiers changes the meaning of the statement.\footnote{It's ok to a bit vague in your answers to this question.}
	\begin{enumerate}
		\item \begin{enumerate}
			\item \emph{For every} even integer $a$ \emph{there exists} an integer $b$ such that $a = 2b$.
			\item \emph{There exists} an integer $b$ such that \emph{for every} even integer $a$, $a = 2b$.
		\end{enumerate}
		\item \begin{enumerate}
			\item \emph{For every} positive integer $n$ \emph{there exists} a positive real number $\epsilon$ such that $ \epsilon < 1/n$.
			\item \emph{There exists} a positive real number $\epsilon$ such that \emph{for every} positive integer $n$, $ \epsilon < 1/n$.\footnote{Such an $\epsilon$ is called an {\bf infinitesimal}.	There are number systems, for example the \textbf{hypperreal numbers}, which extend the real numbers by incorporating infinitesimals.}
		\end{enumerate}
		\item \begin{enumerate}
			\item \emph{For every} even integer $a$, \emph{for every} odd integer $b$, $a + b$ is odd.
			\item \emph{For every} odd integer $b$, \emph{for every} even integer $a$, $a + b$ is odd.
			\end{enumerate}
		\item Let $f: \R \rightarrow \R$ be a function and let $\epsilon$ be a positive real number. \begin{enumerate}
			\item {\it For every} $x$, {\it there exists} a $\delta > 0$ such that {\it for every} $y$, if $|x-y| < \delta$ then $|f(x) - f(y)| < \epsilon$.
			\item {\it There exists} a $\delta > 0$ such that {\it for every} $x$, {\it for every} $y$, if $|x-y| < \delta$ then $|f(x) - f(y)| < \epsilon$.
		\end{enumerate}
	\end{enumerate}
\end{exercise}

	\subsubsection{Negating Nested Quantifiers}
	As we saw in Sections \ref{section:proofByContradiction} and \ref{section:proofByContrapositive} we often need to negate mathematical statements.

	In order to systematically negate complicated logical expressions, we observe that we can negate simple statements in the following manner. (These are called {\it De Morgan's laws}.)
	\begin{center}
		\begin{tabular}{|p{0.475\textwidth}|p{0.475\textwidth}|}
			\hline Statement
				&
					Negation \\ \hline & \\
			For every $x$ the statement $P$ is true.
				&
					There exists an $x$ such that the statement $P$ is false.
					\\& \\ \hline &\\
			There exists an $x$ such that the statement $P$ is true.
				&
					For every $x$ the statement $P$ is false.
					\\& \\ \hline &\\
			$P$ is true and $Q$ is true
				&
					$P$ is false or $Q$ is false.
					\\& \\ \hline &\\
			$P$ is true or $Q$ is true
				&
					$P$ is false and $Q$ is false.
					\\& \\ \hline &\\
			If $P$ then $Q$
				&
					$P$ but not $Q$.
					\\& \\ \hline
	\end{tabular}
	\end{center}
	If we have multiple quantifiers in a statement then we start with the outermost quantifier and recursively move inwards.
	\begin{example}
		The negation of {\it ``For every $x$, there exits a $y$ such that, the statement $P$ is true''} equals {\it ``There exists an $x$ such that, for every $y$, the statement $P$ is false.''}
	\end{example}
	\begin{exercise} Negate each of the following.
		\begin{enumerate}
			\item \emph{For every} even integer $a$ \emph{there exists} an integer $b$ such that $a = 2b$.
			\item \emph{There exists} an integer $b$ such that \emph{for every} even integer $a$, $a = 2b$.
			\item \emph{For every} positive integer $n$ \emph{there exists} a positive real number $\epsilon$ such that $ \epsilon < 1/n$.
			\item \emph{There exists} a positive real number $\epsilon$ such that \emph{for every} integer $n$, $ \epsilon < 1/n$.
			\item \emph{For every} even integer $a$, \emph{for every} odd integer $b$, $a + b$ is odd.
			\item {\it For every} $\epsilon > 0$, {\it for every} $x$, {\it there exists} a $\delta > 0$ such that {\it for every} $y$, if $|x-y| < \delta$ then $|f(x) - f(y)| < \epsilon$.
			\item {\it For every} $\epsilon > 0$, {\it there exists} a $\delta > 0$ such that {\it for every} $x$, {\it for every} $y$, if $|x-y| < \delta$ then $|f(x) - f(y)| < \epsilon$.
		\end{enumerate}
	\end{exercise}

	\noindent {\bf Word of caution:} When the word ``and'' is used in a list it is not considered a logical operator (this is a deficiency of the English language). For example, negating the following expression
	\begin{center}
		\emph{For every} even integer $a$ and \emph{for every} odd integer $b$, $a + b$ is odd,
	\end{center}
	does not change the {\it and} to an {\it or} (what happens if you do change the {\it and} to an {\it or}?).
