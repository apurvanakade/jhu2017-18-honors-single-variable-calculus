%!TEX root=ClassNotes.tex
\section{Trig, Exp, and Log functions}
We'll use the Fundamental Theorem of Calculus to compute the derivatives of trigonometric functions, exponentials, and logarithms.
\subsection{Trigonometric Functions}
In this section, we'll prove that
\begin{align*}
	(\cos x)' &= -\sin x\\
	(\sin x)' &= \cos x
	\end{align*}
We'll only show this for $0 < x < \pi/2$, however,  the statements are true for all real numbers $x$.
We'll do this by first computing $\left(\cos^{-1}x \right)'$ using basic geometry and the Fundamental Theorem and then using the formula for the derivative of inverse functions.

\begin{exercise}
	Consider the unit circle $x^2 + y^2 = 1$. In the following figure, the shaded region is a sector of angle $\theta$ (in radians), for $0 < \theta < \pi/2$, so that the corresponding point on the circle is $(\cos \theta, \sin \theta)$.
	\begin{align*}
		\adjincludegraphics[width=0.3\textwidth, valign=c]{pie1.png}
		=
		\adjincludegraphics[width=0.3\textwidth, valign=c]{pie2.png}
		+
		\adjincludegraphics[width=0.3\textwidth, valign=c]{pie3.png}
	\end{align*}
	\begin{enumerate}
		\item Argue that the above decomposition of areas can be expressed algebraically as
		      \begin{align*}
			      \dfrac{\theta}{2}
			       & =
			      \dfrac{x \sqrt{1 - x^2}}{2}
			      +
			      \int \limits_x^1 \sqrt{1 - t^2} \: dt
		      \end{align*}
		      which can further be simplified to
		      \begin{align*}
			      \dfrac{\cos^{-1}x}{2}
			       & =
			      \dfrac{x \sqrt{1 - x^2}}{2}
			      +
			      \int \limits_x^1 \sqrt{1 - t^2} \: dt
		      \end{align*}

		\item Differentiate both sides to show that
		      \begin{align*}
			      \left(\cos^{-1}x \right)'
			       & =
			      -\dfrac{1}{\sqrt{1-x^2}}
		      \end{align*}

		\item Use the formula for the derivative of inverse function\footnote{$\left(f^{-1}(a)\right)'= \dfrac{1}{f'(f^{-1}(a))}$} to show that
		      \begin{align*}
			      (\cos \theta)' = -\sin \theta
		      \end{align*}
	\end{enumerate}
\end{exercise}


\begin{exercise}
	Differentiate both sides of the trig identity
	\begin{align*}
		\sin^2 x  + \cos^2 x = 1
	\end{align*}
	and use your computation of $(\cos x)'$ to show that
	\begin{align*}
		(\sin x)' = \cos x
	\end{align*}
\end{exercise}

\begin{exercise}
	\begin{enumerate}
		\item Use the formula for the derivative of inverse function to compute the derivative of $\sin ^{-1}x$.
		\item What is the relationship between the derivatives of $\sin ^{-1}x$ and $\cos ^{-1}x$? Why do you think this is the case?
	\end{enumerate}
\end{exercise}

\begin{exercise}
	Compute the derivatives of $\tan x$ and $\tan^{-1} x$.
\end{exercise}


\subsection{Exponential Functions and Logarithms}
Exponential functions are a bit tricky to define from first principles as every natural definition of $e^x$ uses either limits of sequences or differential equations.
We'll instead give a more {\it ad hoc} definition of logarithms as in the book and {\it verify} that it satisfies the properties that logarithms are supposed to satisfy.

\begin{definition}
	Define the {\bf natural logarithm} to be the integral
		\begin{align*}
			\ln x = \int_1^x \dfrac{1}{t} \: dt
		\end{align*}
		for $x > 0$.
\end{definition}
\begin{remark}
	As $\ln x$ is {\it defined} to be the antiderivative of $ 1 / x$,
	\begin{align*}
		\left(\ln x\right)'= \dfrac{1}{x}
	\end{align*}
\end{remark}
\begin{exercise}
	\begin{enumerate}
		\item Draw the graph of $\dfrac{1}{x}$ for $x > 0$.
		\item {\it Geometrically} argue that
			\begin{align*}
				\ln x \mbox{ is }
				\begin{cases}
					\mbox{positive } & \mbox{ if } x > 1\\
					0 & \mbox{ if } x = 1\\
						\mbox{negative } & \mbox{ if } x < 1
				\end{cases}
			\end{align*}
		\item Using u-substitution\hint{In the u-substitution formula $
		\int_a^z f(g(t)) g'(t)\: dt = \int_{g(a)}^{g(z)} f(u) \: du$ use $g(t) = x \cdot t$.} show that
		\begin{align*}
			\int_1^y \dfrac{1}{t} \: dt
			&=
			\int_x^{xy} \dfrac{1}{t} \: dt
		\end{align*}
		for real numbers $x , y > 0$. (You should think about what this means geometrically.)
		\item Show that this implies that
		\begin{align}
			\label{eq:log_identity}
			\ln x + \ln y = \ln xy
		\end{align}
	\end{enumerate}
\end{exercise}
This last identity \eqref{eq:log_identity} is the fundamental identity of logarithms.

\begin{definition}
	Define $\exp(x)$ to be the inverse function of $\ln x$ i.e.
	\begin{align*}
		\exp(\ln x) &= x \\
		\ln(\exp x) &= x.
	\end{align*}
\end{definition}
\begin{definition}
	Define $e$ to be the value of $\exp(x)$ at $x = 1$.
\end{definition}


\begin{exercise}
	\begin{enumerate}
		\item Show that $\exp(0) = 1$.
		\item Show that Equation \eqref{eq:log_identity} implies that
		\begin{align*}
			\exp(a + b) = \exp(a) \cdot \exp(b)
		\end{align*}
		\item Use this to argue that
		\begin{align*}
			\exp(n) = e^n
		\end{align*}
		where $n$ is a positive integer.
		\item {\bf (Optional)} Extend the above statement first to negative integers and then to rational numbers.
	\end{enumerate}
	It follows then from continuity arguments that
	\begin{align*}
		\exp(x) = e^x
	\end{align*}
	for all real numbers $x$.
\end{exercise}


\begin{exercise}
	Use the formula for the derivative of inverse function to show that
	\begin{align*}
		\left(e^x\right)'=e^x\\
	\end{align*}
\end{exercise}



This completes the computation of derivatives of all the standard functions.

\begin{exercise}
	The Fundamental Theorem of Calculus says that if $f'(x)=g(x)$ then $f(x) = \int g(x)\: dx + c $, where $\int g(x)\: dx$ stands for the indefinite integral.
	Go back to your derivative computations in this section and rewrite them as indefinite integrals.
\end{exercise}

We'll next use the u-substitution, and Integration by Parts to compute integrals of functions which can be written in terms of these standard functions.














\newpage
\subsection{Trigonometric Identities}
We'll need several trigonometric identities for computing integrals.
While it is possible to derive these identities using Euclidean geometry, there is a much faster trick to derive these using {\it Euler's identity}.


\subsubsection{Complex Numbers}
First we need some basic algebraic facts about complex numbers. {\bf Complex numbers} are numbers of the form
\begin{align*}
	z = a + b i
\end{align*}
where $a$ and $b$ are real numbers and $i$ is a {\it formal} variable that satisfies
\begin{align*}
	i^2 = -1.
\end{align*} The number $a$ is called the {\bf real part} of $z$, denoted $\mathrm{Re}(z)$, and the number $b$ is called it's {\bf imaginary part}, denoted $\mathrm{Im}(z)$.

A real number $r$ can be thought of the complex number $ r + 0 i$. Thus the set of complex numbers contains the set of real numbers.

As with real numbers, we can add, subtract, multiply, and
divide complex numbers.
\begin{example} Multiplying complex numbers:
	\begin{align*}
			(a + bi)(c - di)
			&= a(c - di) + bi(c - di) \\
			&= ac - adi + bci + bd  & \mbox{ as } i^2 = -1\\
			&= (ac + bd) + i(-ad + bc)
	\end{align*}
\end{example}

\begin{exercise}
	\label{q:complex_conjugate}
	Show that
	\begin{align*}
		(a + bi) (a - bi) = a^2 + b^2.
	\end{align*}
\end{exercise}

The complex number $\overline{z} = a - bi$ is called the {\bf complex conjugate} of $z=a + bi$.
By the above exercise, we get a real number after multiplying a complex number by it's conjugate which is an extremely useful fact.

\begin{example}
		Complex conjugates are useful when dividing complex numbers. To simplify a fraction, we multiply and divide by the complex conjugate of the denominator.
		\begin{align*}
			\dfrac{a+bi}{c+di}
			&=
			\dfrac{a+bi}{c+di} \cdot	\dfrac{c - di}{c-di}\\
			&=
			\dfrac{(a+bi)(c - di)}{c^2+d^2} & \mbox{ by Exercise \eqref{q:complex_conjugate}}\\\
			&=
			\dfrac{(ac + bd) + i(-ad + bc)}{c^2+d^2} \\
			&=
			\dfrac{ac + bd}{c^2+d^2}
			+
			i\cdot \dfrac{-ad + bc}{c^2+d^2}
		\end{align*}
\end{example}

\begin{exercise}
	Find the real and imaginary parts of the following complex numbers:
	\begin{multicols}{2}
		\begin{enumerate}
			\item $(2-3i)(i)$
				\item $(2-3i)(1+i)$
					\item $\dfrac{1}{i}$
						\item $\dfrac{1}{1+i}$
							\item $\dfrac{2-3i}{i}$
								\item $\dfrac{2-3i}{1+i}$
		\end{enumerate}
	\end{multicols}
\end{exercise}



\subsubsection{Euler's Identity}
It is possible to extend trigonometric and exponential functions (but not logarithms) to complex numbers.
These functions have the same derivatives and integrals as in the real case.
The following {\bf Euler's Identity} establishes a deep connection between trigonometric and exponential functions.
\begin{theorem}[Euler's identity]
\begin{align*}
		e^{i \theta} = \cos \theta + i \sin \theta
\end{align*}
for all real numbers $\theta$.
\end{theorem}
The proof of this theorem requires us to extend the entire theory of calculus to complex numbers (called complex analysis) and is beyond the scope of this class.
We'll simply use it as a fast method of (re)deriving several trig identities, and later on for doing integrals computations.

\begin{example}
	We know that
	\begin{align*}
		&& e^{i \theta} \cdot e^{-i \theta} &= e^0
	\end{align*}
	Simplifying the left hand side we get,
	\begin{align*}
		\Rightarrow
		&&
		(\cos \theta + i \sin \theta) \cdot (\cos (-\theta) + i \sin (-\theta)) &= e^0  \\
		\Rightarrow
		&&
		(\cos \theta + i \sin \theta) \cdot (\cos \theta - i \sin \theta) &= 1  \\
		\Rightarrow &&
		\cos^2 \theta + \sin^2 \theta
		&= 1 & \mbox{ by \eqref{q:complex_conjugate}}
	\end{align*}
	which is the fundamental identity of trigonometric functions.
	(In the above derivation we used the fact that $\cos(-\theta) = \cos \theta$ and $\sin (-\theta) = - \sin \theta$.)
\end{example}

\begin{exercise}
	Using Euler's identity and
	\begin{align*}
		\left(e^{i \theta}\right)^2 = e^{2i \theta}
	\end{align*}
	find the formulae for $\cos 2 \theta$ and $\sin 2 \theta$. (These are called the {\bf double angle formulae}.)
\end{exercise}

\begin{exercise}
	\begin{enumerate}
		\item
		Using Euler's identity and
		\begin{align*}
			e^{i x} \cdot e^{i y}  = e^{i (x + y)}
		\end{align*}
		find the formulae for $\cos (x + y)$ and $\sin (x + y)$.
		\item
		Using Euler's identity and
		\begin{align*}
			e^{i x} \cdot e^{-i y}  = e^{i (x - y)}
		\end{align*}
		find the formulae for $\cos (x - y)$ and $\sin (x - y)$.
	\end{enumerate}
\end{exercise}


\begin{exercise}
	Using the previous exercise, show that
 \begin{align*}
		 2 \cos x \cos y &= \cos (x-y) + \cos(x+y) \\
	 2 \sin x \sin y &= \cos (x-y) - \cos(x+y)\\
		2 \sin x \cos y &= \sin (x-y) + \sin(x+y)
 \end{align*}
\end{exercise}

We'll now apply these identities to compute some definite integrals.
In later sections, we'll use these to compute indefinite integrals  such as $\int e^{ax} \sin {bx} \: dx$.

\begin{exercise}
	Let $n$ be an integer.
	\begin{enumerate}
		\item Draw graphs of
		\begin{align*}
			\sin x, \sin (-x), \sin 2x, \sin (-2x), \\
			\cos x, \cos (-x), \cos 2x, \cos (-2x),
		\end{align*} for $ -\pi \le x \le \pi$.
		\item Geometrically argue that
		\begin{align*}
			\int_{-\pi}^{\pi} \cos (nt) \: dt &=
			\begin{cases}
				0 & \mbox{ if } n \neq 0 \\
				2 \pi & \mbox{ if } n = 0
			\end{cases}
		\end{align*}
		\item Geometrically argue that
		\begin{align*}
			\int_{-\pi}^{\pi} \sin (nt) \: dt &= 0
		\end{align*}
		\item Why do we need to make two cases for $\cos x$ but not for $\sin x$?
	\end{enumerate}
\end{exercise}

\begin{exercise}
	Let $m$ and $n$ be positive integers. Using the previous two exercises, show that
		\begin{align*}
			 \int_{-\pi}^{\pi} \cos (nt) \cos (mt) \: dt
			 &=
			 \begin{cases}
			 	0 & \mbox{ if } m \neq n \\
 			 	\pi & \mbox{ if } m = n
			 \end{cases} \\
			 \int_{-\pi}^{\pi} \sin (nt) \sin (mt) \: dt
			 &=
			 \begin{cases}
				0 & \mbox{ if } m \neq n \\
				\pi & \mbox{ if } m = n
			 \end{cases} \\
			 \int_{-\pi}^{\pi} \cos (nt) \sin (mt) \: dt
			 &=
			 0
		\end{align*}
\end{exercise}
These integrals are the foundational identities for Fourier analysis, which is one of the most powerful mathematical theories emerging out of elementary calculus.


% \begin{exercise} Verify that
% 	\begin{align*}
% 			\cos \theta &= \dfrac{e^{i \theta} + e^{-i \theta}}{2} \\
% 					\sin \theta &= \dfrac{e^{i \theta} - e^{-i \theta}}{2i}
% 	\end{align*}
% \end{exercise}
%
% \begin{exercise}
% 	Compute the indefinite integrals
% 	\begin{align*}
% 		\int
% 	\end{align*}
% \end{exercise}
