%!TEX root=ClassNotes.tex

\section{Computing Indefinite Integrals}

In this section, we're interested in computing {\it indefinite integrals} or {\it antiderivatives}.
Indefinite integrals cannot be computed for all functions, for example, there is no closed form for the {\bf Gaussian Integral}
\begin{align*}
	\int \exp\left(-\dfrac{x^2}{2} \right) \: dx
\end{align*}
which is widely used in statistics, and the only way to compute it is using numerical methods.


There is no general method of computing indefinite integrals.
Instead, in this section we'll learn a few common tricks which you can combine in various ways to solve complicated problems.
There are two basic tools for computing indefinite integrals
\begin{enumerate}
	\item u-substitution and
	\item integration by parts.
\end{enumerate}
These are often combined with various algebraic and trig identities.

One thing that makes the computation of indefinite integrals far more complicated than that of derivatives, in practice, is that similar looking functions can have wildly different integrals. For example,
\begin{align*}
	\int 1/x \: dx              & = \ln x  + c                          \\
	\int 1/x^2 \: dx            & = -1/x + c                            \\
	\int 1/{(1+x^2)} \: dx      & = \tan^{-1} x + c                     \\
	\int 1/(1 - x^2) \: dx      & = 1/2 \cdot \ln ((1 + x)/(1 - x)) + c \\
	\int 1/\sqrt{1 - x^2} \: dx & = \sin^{-1} x + c
\end{align*}
So just by looking at a problem you cannot easily {\it guess} what it's integral is going to look like.

\subsection{u-substitution}
In terms of indefinite integrals, u-substitution takes the form
\begin{align}
	\label{eq:u-sub}
	\int f\left(g(x)\right) g'(x)\: dx = \int f(u) \: du + \mathrm{constant}
\end{align}
Ideally we want the integral on the right-hand side to be simpler than the one on the left-hand side.
We interpret this identity as having obtained the right-hand from the left-hand side by making the ``substitution''
\begin{align*}
	u  & = g(x)        \\
	du & = g'(x) \: dx
\end{align*}
This is just a {\it mnemonic} and doesn't have a concrete mathematical meaning (at least not a simple one).

\begin{remark}
	After finding the integral $\int f(u) \: du$ you should ALWAYS plug back $u = g(x)$ to get your final answer in terms of the original variable.
\end{remark}

\begin{exercise}
	Find the following indefinite integrals using u-substitution(s).
	\begin{multicols}{2}
		\begin{enumerate}
			\item $\int x e^{x^2} \: dx $
			\item $\int \dfrac{1}{x \ln x} \: dx$
			\item $\int \dfrac{e^{\sqrt{x}}}{\sqrt{x}} \: dx$
			\item $\int \dfrac{x}{4x^2+5} \: dx$
			      % \item $\int \dfrac{x}{2x^2 - 3} \: dx$
			\item $\int \dfrac{1}{x^2 + 2} \: dx$
			\item $\int \sec^2 x \tan^5 x \: dx$
			\item $\int \tan x \: dx$
			\item $\int \dfrac{e^x}{e^{2x} + 2e^x + 1} \: dx$
		\end{enumerate}
	\end{multicols}
\end{exercise}

% \begin{exercise}
%   Sometimes there's no obvious reason to use u-substitution. Instead, we use u-substitution in the hope that integral simplifies.
%   \begin{enumerate}
% \item $\dfrac{e^x}{1 - e^{2x}}$
%     \item $\dfrac{1 - e^x}{1 + e^{x}}$
%     \item $\dfrac{e^{2x}}{\sqrt{1 + e^x}}$
%   \end{enumerate}
% \end{exercise}

\subsection{Integration by Parts}
Integration by parts is {\it sometimes} used to integrate products of functions (you should always try u-substitution first, unless the use of integration by parts is extremely obvious).
There are various ways of memorizing integration by parts, you should pick one that you find easy to remember.

One possible way to express it is as follows.
\begin{align}
	\int fg  =  fG - \int f' G  + \mathrm{constant}
\end{align}
where $G$ is any antiderivative of $g$.
In most problems, integration by parts is useful only if the function $f'G$ is simpler that the function $fg$.

\begin{example}
	For computing $\int x e^x \: dx$ by parts, we have (at least) two possible choices:
	\begin{enumerate}
		\item $f = x \mbox{ and } g = e^x$.
		      In this case,
		      \begin{align*}
			      f' G = e^x
		      \end{align*}
		\item $f = e^x \mbox{ and } g = x$.
		      In this case,
		      \begin{align*}
			      f' G = e^x \cdot x^2/2
		      \end{align*}
		      In the first case, the function $f' G$ is simpler than $fg$ and hence this is the decomposition we should use for applying integration by parts.
	\end{enumerate}
\end{example}

\begin{exercise}
	Find the following indefinite integrals using integration by parts.
	\begin{multicols}{2}
		\begin{enumerate}
			\item $\int x e^x\: dx$
			\item $\int x^2 e^{x}\: dx$
			\item $\int x \sin x\: dx$
			\item $\int \ln x \: dx$ \hint{ Use $f = \ln x$ and $g = 1$.}
			\item $\int x \ln x\: dx$
			\item $\int x (\ln x)^2\: dx$
		\end{enumerate}
	\end{multicols}
\end{exercise}


\subsection{Trigonometric Integrals}
There are a LOT of tricks for computing integrals of functions involving trigonometric functions.
We'll only compute integrals of two kinds of functions:
\begin{align*}
	\sin^ m x \cos^n x \mbox{ \qquad and \qquad } e^{ax} \sin bx
\end{align*}

Let $m$ and $n$ be non-negative integers.
For integrating $\sin^ m x \cos^n x$ if
	{\bf either $m$ or $n$ is odd}
then we get the answer by a simple u-substitution. For example,
\begin{align*}
	\sin^ {2k+1} x \cdot \cos^n x
	 & = \sin^ {2k} x \cdot  \sin x \cdot  \cos^n x                 \\
	 & = \left( \sin^2 x \right)^k \cdot \sin x \cdot \cos^n x      \\
	 & = \left( 1 - \cos^2 x \right)^k \cdot \sin x \cdot  \cos^n x
\end{align*}
after which we can integrate using the u-substitution $ u = \cos x$.
\begin{exercise}
	Compute the following indefinite integrals.
	\begin{multicols}{2}
		\begin{enumerate}
			\item $\int \sin^3 x \cos^2 x \: dx$
			\item $\int \sin^2 x \cos^5 x \: dx$
		\end{enumerate}
	\end{multicols}
\end{exercise}

Let $m$ and $n$ be non-negative integers.
For integrating $\sin^ m x \cos^n x$ if
	{\bf both $m$ and $n$ are even}
then we use the double angle formulae,
\begin{align*}
	\sin^ {2k} x
	 & = \left(\sin^{2} x \right)^k
	 &                                           &  &
	\cos^ {2l} x
	 & = \left(\cos^{2} x \right)^l                   \\
	\\
	 & = \left(\dfrac{1 - \cos(2x)}{2} \right)^k
	 &                                           &  &
	 & = \left(\dfrac{1 + \cos(2x)}{2} \right)^l
\end{align*}
\begin{exercise}
	Compute the following indefinite integrals.
	\begin{multicols}{2}
		\begin{enumerate}
			\item $\int \sin^2 x \: dx $
			\item $\int \sin^2 x \cos^2 x \: dx$
		\end{enumerate}
	\end{multicols}
\end{exercise}

Let $a$, $b$ be real numbers.
Using the Euler's identity
\begin{align*}
	e^{ibx} = \cos bx + i \sin bx
\end{align*}
we get
\begin{align*}
	\int e^{ax} \cos bx \: dx & = \mbox{real part of } \int e^{ax} e^{ibx} \: dx      \\
	\int e^{ax} \sin bx \: dx & = \mbox{imaginary part of } \int e^{ax} e^{ibx} \: dx\end{align*}
\begin{exercise}
	Compute the following indefinite integrals.
	\begin{multicols}{2}
		\begin{enumerate}
			\item $\int e^{ax} \cos bx \: dx$
			\item $\int e^{ax} \sin bx \: dx $
		\end{enumerate}
	\end{multicols}
\end{exercise}




\subsection{Trigonometric Substitutions}

Trigonometric substitutions are extremely useful when {\it eliminating radicals} (among other things) because of the identities
\begin{align}
	\begin{split}
		\label{eq:trig_identities}
		\sin^2 x + \cos^2 x &= 1 \\
		\sec^2 x &= 1 + \tan^2 x
	\end{split}
\end{align}

\begin{example}
	If there is a term
	\begin{align*}
		\sqrt{1 + x^2}
	\end{align*}
	in our integral, then we can substitute $ x = \tan u$ so that
	\begin{align*}
		\sqrt{1 + x^2}
		 & = \sqrt{1 + \tan^2 u}                                         \\
		 & = \sec u              & \mbox{ by \eqref{eq:trig_identities}}
	\end{align*}
\end{example}

We'll need a few preliminary computations.
\begin{exercise}
	\begin{enumerate}
		\item Compute $(\sec x)'$.
		\item Show using the fundamental theorem of calculus that
		      \begin{align*}
			      \int \sec x \: dx & = \ln (\sec x + \tan x) + c  \\
			      \int \csc x \: dx & = -\ln (\csc x + \cot x) + c
		      \end{align*}
		\item {\bf Optional: } Compute the integrals $\int \sec x \: dx$ and $\int \csc x \: dx$ directly (without using the fundamental theorem).
	\end{enumerate}
\end{exercise}

\begin{exercise}
	Compute the following integrals using trig substitutions.
	\begin{multicols}{2}
		\begin{enumerate}
			\item $\int \dfrac{1}{\sqrt{1-x^2}}\: dx$
			\item $\int \dfrac{1}{\sqrt{1+x^2}} \: dx$
			\item $\int \dfrac{1}{\sqrt{x^2-1}} \: dx$
			\item $\int \dfrac{1}{x\sqrt{x^2-1}} \: dx$
			\item $\int \dfrac{1}{x\sqrt{1-x^2}} \: dx$
			\item $\int \dfrac{1}{x\sqrt{1+x^2}} \: dx$
			\item $\int \sqrt{1-x^2}\: dx$
			\item $\int x^3 \sqrt{1-x^2}\: dx$
		\end{enumerate}
	\end{multicols}
\end{exercise}

\begin{exercise}{{\bf (Optional)}}
	\begin{enumerate}
		\item Compute the integral
		      \begin{align*}
			      \int \sec^3 x \: dx
		      \end{align*}
		\item Compute the integral
		      \begin{align*}
			      \int \sqrt{1 + x^2} \: dx
		      \end{align*}
	\end{enumerate}
\end{exercise}






\subsection{Partial Fractions}
Partial fractions is a technique used to compute integrals of the form
\begin{align*}
	\int \dfrac{P(x)}{Q(x)} \: dx
\end{align*}
where $P(x)$ and $Q(x)$ are polynomials.
The higher the degree of the denominator $Q(x)$ the harder it is to compute the integral.\\

\subsubsection*{Linear Polynomials}
When the denominator $Q(x)$ is linear the integral can be computed easily using u-substitution.
\begin{exercise}
	Compute the following integrals
	\begin{multicols}{2}
		\begin{enumerate}
			\item $ \int \dfrac{x^2+1}{x} \: dx$
			\item $ \int \dfrac{x^2 + 1}{x+1} \: dx$
			\item $ \int \dfrac{x}{2x-3} \: dx$
			\item $ \int \dfrac{x^2}{2x+2} \: dx$
		\end{enumerate}
	\end{multicols}
\end{exercise}

\subsubsection*{Quadratic with Repeated Roots}
When the denominator is a quadratic $Q(x) = ax^2 + bx + c$ there are 3 different methods for finding the integral, depending on what the roots of $Q(x)$ are.
\begin{align*}
	x^2 + ax + b \mbox{ has }
	\begin{cases}
		\mbox{ repeated roots } \\
		\mbox{ complex roots }  \\
		\mbox{ real non-repeated roots }
	\end{cases}
\end{align*}

The first case of {\bf repeated roots} is the easiest. In this case, our goal is to find a simple u-substitution to reduce the problem to
\begin{align*}
	\int \dfrac{R(u)}{u^2} \: du
\end{align*}
where $R(u)$ is some polynomial.
\begin{exercise}
	For each of the following problems, verify that the denominator has repeated roots. Then
	compute the integrals.
	\begin{multicols}{2}
		\begin{enumerate}
			\item $ \int \dfrac{x}{4x^2 - 4x + 1} \: dx$
			\item $ \int \dfrac{x^2}{x^2 + 2x + 1} \: dx$
		\end{enumerate}
	\end{multicols}
\end{exercise}

\subsubsection*{Quadratic with Complex Roots}
When the denominator $Q(x) = ax^2 + bx + c$ has complex roots our goal is to find a u-substitution to reduce the problem to the integrals
\begin{align*}
	\int \dfrac{1}{u^2 + 1} \: du
	 &  & \mbox{ and }
	 &  &
	\int \dfrac{u}{u^2 + 1} \: du
\end{align*}

\begin{exercise}
	For each of the following problems, verify that the denominator has complex roots. Then
	compute the integrals.
	\begin{multicols}{2}
		\begin{enumerate}
			\item $ \int \dfrac{1}{x^2 + 4} \: dx$
			\item $ \int \dfrac{3}{x^2 + 2x + 2} \: dx$
			\item $ \int \dfrac{3x}{x^2 + 2x + 2} \: dx$
			\item $ \int \dfrac{x}{4x^2 - 4x + 3} \: dx$
		\end{enumerate}
	\end{multicols}
\end{exercise}

\begin{exercise}If the degree of the numerator is $ \ge 2$ then we first have to do a long division to simplify the numerator. Compute the following integrals.
	\begin{multicols}{2}
		\begin{enumerate}
			\item $ \int \dfrac{x^2}{x^2 + 4} \: dx$
			\item $ \int \dfrac{3x^3}{x^2 + 2x + 2} \: dx$
		\end{enumerate}
	\end{multicols}
\end{exercise}



\subsubsection*{Quadratic with Real Non-repeated Roots}
Finally, when the denominator $Q(x)$ has real non-repeated roots we need to use the method of partial fractions.\\


In this method we first need to find a factorization of $Q(x)$ as a product of linear terms, say $Q(x) = Q_1(x) \cdot Q_2(x)$ where both $Q_1(x)$ and $Q_2(x)$ are linear. If the degree of the numerator $P(x)$ is $\le 1$ then we can always write
\begin{align*}
	\dfrac{P(x)}{Q(x)}
	 & =
	\dfrac{A}{Q_1(x)} + \dfrac{B}{Q_2(x)}
\end{align*}
for some constants $A$ and $B$.
This is called the {\bf partial fraction decomposition} of $\frac{P(x)}{Q(x)}$. We find $A$ and $B$ by multiplying both sides by $Q(x)$ and comparing the coefficients on the left and right hand sides.


\begin{exercise}
	For each of the following problems, verify that the denominator has real non-repeated roots. Then
	compute the integrals.
	\begin{multicols}{2}
		\begin{enumerate}
			\item $ \int \dfrac{1}{x^2 - 1} \: dx$
			\item $ \int \dfrac{4x}{x^2 - 4} \: dx$
			\item $ \int \dfrac{5}{x^2 - 2x} \: dx$
			\item $ \int \dfrac{3 x}{x^2 + x - 2} \: dx$
		\end{enumerate}
	\end{multicols}
\end{exercise}

\begin{exercise}As before if the degree of the numerator is $ \ge 2$ then we first need to do long division to simplify the numerator. Compute the following integrals.
	\begin{multicols}{2}
		\begin{enumerate}
			\item $ \int \dfrac{3 x^2}{x^2 + x - 2} \: dx$
			\item $ \int \dfrac{x^2 - 1}{x^2 - 4} \: dx$
			\item $ \int \dfrac{4x^3 - 3x + 5}{x^2 - 2x} \: dx$
		\end{enumerate}
	\end{multicols}
\end{exercise}


\subsubsection*{Degree $\ge 3$}
When degree of the denominator $Q(x)$ is $\ge 3$, one can show the existence of a partial fraction decomposition using {\it abstract algebra}.
But the details are much more complicated and hard to do by hand.

Instead, feel free to use the internet to compute partial fraction decompositions. For example, go to \url{http://www.wolframalpha.com} and input
\begin{verbatim}
	partial fractions (x^2 - 2)/(x+1)^3
\end{verbatim}
for computing the partial fraction decomposition of $\dfrac{x^2 - 2}{(x+1)^3}$.

Once you have the partial fraction decomposition you can use u-substitution to compute the integral.

\begin{exercise}
	Compute the following integrals using partial fraction decompositions.
	\begin{multicols}{2}
		\begin{enumerate}
			\item $ \int \dfrac{x^2 - 2}{(x+1)^3} \: dx$
			\item $ \int \dfrac{8}{3x^3 + 7x^2 + 4x} \: dx$
			\item $ \int \dfrac{x^3 + 8}{(x^2 - 1)(x - 2)} \: dx$
		\end{enumerate}
	\end{multicols}
\end{exercise}











\newpage
\subsection{Practice Problems}
We've learned several techniques for computing indefinite integrals. Of these
\begin{enumerate}
	\item Basic u-substitution
	\item Trigonometric substitutions (+ trig identities)
\end{enumerate}
are the tricky ones as there are a lot of possible substitutions to choose from. The other three
\begin{enumerate}[resume]
	\item Integration by Parts
	\item Trigonometric integrals
	\item Partial fractions
\end{enumerate}
are much easier to use.\\

 The following problems will require you to use all of the above techniques. You should not expect to {\it see} the solution right away, instead, systemically try different things until you reduce the problem to something that looks familiar.

\begin{exercise}
	Compute the following integrals.
	\begin{multicols}{2}
		\begin{enumerate}
			\item $\int \dfrac{e^x}{(e^x - 1)(e^x - 3)}\: dx$
			\item $\int \dfrac{1}{1 + e^x} \: dx$
			\item $\int e^{\sqrt{x}} \: dx$
			\item $\int \sin \sqrt{x}\: dx$
			\item $\int \dfrac{\sin^3 x}{\cos^2 x} \: dx$
			\item $\int \dfrac{1 - \sin x}{\cos^2 x}\: dx$
			\item $\int \dfrac{1}{1 + \sin x}\: dx$
			\item $\int \sqrt{1 + \cos 2x}\: dx$
			\item $\int \sec^3 x \tan x\: dx$
			\item $\int x \tan^{-1} x\: dx$
			\item $\int x^2 \tan^{-1} x\: dx$
			\item $\int \tan^{-1} {\sqrt{x}}\: dx$
			\item $\int \dfrac{x}{\sqrt{2 + 2x + x^2}} \: dx$
			\item $\int \dfrac{1}{\sqrt{2x - x^2}}\: dx$
			\item $\int \ln(1 + x^2)\: dx$
			\item $\int \tan^{2} x\: dx$
		\end{enumerate}
	\end{multicols}
\end{exercise}
