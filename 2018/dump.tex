
% 	\begin{exercise}
% 		\begin{align*}
% 			\sqrt{1 - x^2}
% 		\end{align*}
% 	\end{exercise}



\section*{Solutions to Selected Problems}

\begin{exercise*}
  Find
  $\lim \limits_{x \rightarrow 0} f(x)$ where $$f(x) = \begin{cases}
            x & \mbox{ if $x$ is rational}   \\
            0 & \mbox{ if $x$ is irrational}
          \end{cases}$$
\end{exercise*}
\begin{proof}
  We'll prove that $\lim \limits_{x \rightarrow 0} f(x) = 0$. For this we need to show that for every $\epsilon$, there exists a $\delta$ such that for every $x$, if $0 < |x| < \delta$ then $|f(x)| < \epsilon$.

  Let $\epsilon > 0$, let $\delta = \epsilon$ and let $0 < |x| < \delta$ then either $x$ is a rational number or it is an irrational number.

  If $x$ is a rational number then
  \begin{align*}
    |f(x)| = |x| < \delta = \epsilon
  \end{align*}
  If $x$ is an irrational number then
  \begin{align*}
    |f(x)| = 0 < \epsilon
  \end{align*}
  which shows that $|f(x)|< \epsilon$ for all $0 < |x| < \delta$, which is what we wanted to prove.
\end{proof}

\begin{exercise*}
  Let $
    f(x) = \begin{cases}
      1 & \mbox{if $x$ is rational,}   \\
      0 & \mbox{if $x$ is irrational.}
    \end{cases}
  $
  \begin{enumerate}
    \item Prove that for every real number $L$, $\lim \limits_{x \rightarrow 0} f(x) \neq L$.
    \item Prove that $\lim \limits_{x \rightarrow 0} f(x) \neq \infty$. (Similarly for $-\infty$.)
  \end{enumerate}
  Hence the limit of $f$ at $0$ does not exist.
\end{exercise*}
\begin{proof}
  \begin{enumerate}
    \item   For the first part, we need to show that for any real number $L$, there exists an $\epsilon > 0$ such that for every $\delta > 0$ there exists an $x$ such that $0 < |x| < \delta$ and $|f(x) - L | \ge \epsilon$. We'll show that this is true for $\epsilon = 1/2$.

    Let $\epsilon = 1/2$, let $\delta > 0$, let $x_1$ be an irrational number satisfying $0 < |x_1| < \delta$ and let $x_2$ be a rational number satisfying $0 < |x_2 | < \delta$.

    We have already shown that for any $L$ at least one of $|L|$ or $|1 - L|$ is $\ge 1/2$. Hence either \begin{align*}
      |f(x_1) - L| \ge 1/2 \quad \mbox{ or } \quad |f(x_2) - L | \ge 1/2
    \end{align*}
    Thus the inequality $|f(x) - L | \ge \epsilon$ is true for either $\epsilon = x_1$ or $x_2$, which completes the proof.

    \item For the second part, we need to show that there exists an $\epsilon > 0$ such that for every $\delta > 0$ there exists an $x$ such that $0 < |x| < \delta$ and $|f(x)| \le \epsilon$. We'll show that this is true for $\epsilon = 1$.

    Let $\epsilon = 1$, let $\delta > 0$, let $x$ be an irrational number satisfying $0 < |x| < \delta$.

    Then $|f(x)| = 0 \le 1 = \epsilon$ which is what we wanted to prove.

    Similar proof works for $-\infty$.\\
  \end{enumerate}
  Since $\lim \limits_{x \rightarrow 0} f(x)$ does not equal any real number, $\infty$, or $-\infty$, the limit of $f$ at 0 does not exist.
\end{proof}
