\documentclass[9pt, a4paper, oneside]{amsart}


\usepackage{enumitem}
\usepackage{parskip}
\usepackage{fancyhdr}
\usepackage{color}
\usepackage{multicol}
\pagestyle{fancy}

\newlist{questions}{enumerate}{1}
\setlist[questions, 1]{label = \bf Q.\arabic*., itemsep=1em}


% \setlength{\topmargin}{-10mm}
% \setlength{\textheight}{235mm}
% \setlength{\oddsidemargin}{-3mm}
% \setlength{\textwidth}{165mm}
% \setlength{\footskip}{10mm}






\lhead{\scshape Apurva Nakade}
\rhead{\scshape Honors Single Variable Calculus}
\renewcommand*{\thepage}{\small\arabic{page}}
\title{Problem Set 03}

\begin{document}

\maketitle
\thispagestyle{fancy}


\section*{Part 1 - Continuity}

\begin{questions}[resume]

	\item
	\begin{enumerate}
		\item Use the $ \epsilon - \delta$ definition to show that the constant function $ f(x) = c $ is continuous everywhere.

		\item Use the $ \epsilon - \delta$ definition to show that the function $ f(x) = x $ is continuous everywhere.

		\item Let $ p(x)$ a polynomial of degree $ n$. Use Theorem 2 to prove that $p(x)$ is continuous everywhere.
	\end{enumerate}

	Trigonometric functions, exponential functions and logarithms are also continuous wherever they are defined. We will assume this fact without proof for now and perhaps come back to it later.

	\item
	\begin{enumerate}
		\item Prove that if $ \lim \limits_{x \rightarrow 0} f(x)/x = l$ and $ b \neq 0$ then $$ \lim \limits_{x \rightarrow 0} f(bx)/x = bl$$
		\item Assuming $ \lim \limits_{x \rightarrow 0} \sin x / x = 1$ find  $ \lim \limits_{x \rightarrow 0}$ for each of the following functions,
		      \begin{multicols}{3}
		      	\begin{enumerate}
		      		\item $\sin (2x)/ x$
		      		\item $\sin (ax)/ \sin(bx)$
		      		\item $\sin^2 (2x)/ x$
		      		\item $\sin^2 (2x)/ x^2$
		      	\end{enumerate}
		      \end{multicols}

	\end{enumerate}


	\item Determine, with proof, the points at which the following function is continuous.
	$$ f(x) = \begin{cases} 0 & \mbox{ if $ x$ is rational } \\ x & \mbox{ otherwise }\end{cases}$$


	\item Suppose that $ g$ is continuous on $[a,b]$ and $ h$ is continuous on $[b,c]$ and that $ g(b) = h(b)$. Define a new function
	\begin{align*}
		f(x) = \begin{cases} g(x) & \mbox{ if }a \le x < b \\ h(x) & \mbox{ if }b \le x \le c  \end{cases}
	\end{align*}
	Show that $ f(x)$ is continuous on $ [a,c]$. (Thus, continuous functions can be `glued'.)

\end{questions}










\newpage
\section*{Part 2 - Three Hard Theorems}
In this HW you can use any of the Theorems 1-9 from Chapter 7 to solve the problems.
\begin{questions}[resume]
	\item For each of the polynomial functions $ f$, find (by trial and error) an integer $ n$ such that $ f(x) = 0$ for some $ x$ in  $[n,n+1]$.
	\begin{multicols}{2}
		\begin{enumerate}
			\item $ x^3 - x + 3$
			\item $ x^5 + 5x^4 + 2x + 1$
			\item $ x^5 + x +1$
			\item $ 4x^2 - 4x + 1$
		\end{enumerate}
	\end{multicols}

	\item Suppose $ f$ and $ g$ are continuous on $ [a,b]$ and that $ f(a) < g(a)$ and $ f(b) > g(b)$. Prove that $ f(x) = g(x)$ for some $ x$ in $ (a,b)$.

	\item Suppose that $ f$ is a continuous function with $ f(x) > 0$ for all $ x$, and $ \lim \limits_{x \rightarrow \infty} f(x) = 0 = \lim \limits_{x \rightarrow -\infty} f(x)$. (Draw a picture.) We want to prove that there is some number $ y$ such that $ f(y) \ge f(x)$ for all $ x$.
	\begin{enumerate}
		\item Choose an arbitrary positive real number $s$. Using the $ \epsilon - \delta $ definition of $ \lim \limits _ {x \rightarrow \infty}$ and $ \lim \limits _ {x \rightarrow -\infty}$ show that there is a number $ N$ such that $f(s) > f(x)$ for all $ |x| > N$.
		\item Let $ M = \max(N,s)$ and let $ y$ be the real number in the the interval $ [-M,M]$ such that $ f(y) \ge f(x)$ for all $ x$ in $[-M,M]$. Why does such a $ y$ exist?
		\item Using the fact that $ s \le M$ show that $ f(y) > f(x)$ for all $ |x| > M$.
		\item  Using part (2) and (3) conclude that $ f(y) \ge f(x)$ for all real numbers $ x$.

		\item (Optional) Does there exist a real number $ y$ such that $ f(y) \le f(x)$ for all $ x$?
	\end{enumerate}

	\item Suppose that $ f(x)$ is continuous on $ [a,b]$ and that $ f(x)$ is always rational. What can be said about $ f$? (Hint: Use the intermediate value theorem.)

\end{questions}










\newpage
\section*{Part 3 - Least Upper Bounds}
\begin{questions}[resume]
	\item In this problem we'll prove the following theorem.
	\begin{quote}
		\textbf{Theorem 9.} For every polynomial $ p(x)$ of odd degree $ n$ there is a real number $ x$ such that $ p(x)=0$.
	\end{quote}
	The idea of the proof is to show that for large values of $|x|$ the polynomial $ p(x)$ behaves like $ x^n$. Without any loss of generality assume that the polynomial is of the form
	\begin{align*}
		p(x) & = x^n + a_{1} x^{n-1} + a_{2} x^{n-2} + \cdots + a_{n-1} x + a_n                                                      \\
		     & = x^n\left( 1 + \dfrac{a_{1}} {x} + \dfrac{a_{2}} {x^2} + \cdots + \dfrac{a_{n-1}}{x^{n-1}} + \dfrac{a_n}{x^n}\right)
	\end{align*}
	\begin{enumerate}
		\item Argue that there exists a number $ N$ such that $ \left|\dfrac{a_{i}}{x^{i}}\right| < \dfrac{1}{2n}$ for all $ |x| > N$ and for all $1 \le i \le n$. \\(It's ok to not be completely rigorous for this part.)
		\item Show that for all $ |x| > N$,
		      \begin{align*}
		      	-\dfrac{1}{2} < \dfrac{a_{1}} {x} + \dfrac{a_{2}} {x^2} + \cdots + \dfrac{a_{n-1}}{x^{n-1}} + \dfrac{a_n}{x^n} < \dfrac{1}{2}
		      \end{align*}
		\item
		      Show that for all $ |x| > N$,
		      \begin{align*}
		      	\dfrac{1}{2} < \dfrac{p(x)}{x^n} < \dfrac{3}{2}
		      \end{align*}
		\item Argue that because $ n$ is odd this implies that there exist real numbers $ a > N$, $b < -N$ with $ p(a) > 0 > p(b)$. Conclude that $ p(x) = 0$ for some number $ x$.
	\end{enumerate}

	\item
	Find the supremum (least upper bound) and infimum (greatest lower bound) of the following sets. (Proofs are not required, it's enough to draw pictures.)
	\begin{multicols}{2}
		\begin{enumerate}
			\item $ \left\{ 1/n : n \in \mathbb{N} \right\}$
			\item $ \left\{ 1/n : n \in \mathbb{Z} \mbox{ and } n \neq 0 \right\}$
			\item $ \left\{ 1/n + (-1)^n: n \in \mathbb{N} \right\}$
			\item $ \left\{ x : x^2 < 2 \mbox{ and } x \mbox{ is rational} \right\}$
			\item $ \left\{ x : x^2 + x - 2 \ge 0  \right \}$
			\item $ \left\{ x : x^2 + x - 2 \le 0  \right \}$
		\end{enumerate}
	\end{multicols}
	%
	% \item For a non-empty set $ A$ let $ -A$ denote the set of all $ -x$ for $ x$ in $ A$. Prove that
	% \begin{enumerate}
	% 	\item $ \sup (-A) = - \inf A$
	% 	\item $ \inf (-A) = - \sup A$
	% \end{enumerate}

	\item In this problem we'll prove the Intermediate Value Theorem using (P13).
	\begin{quote}
		\textbf{Intermediate Value Theorem.} If $ f$ is a continuous function on $ [a,b]$ satisfying $ f(a) < c < f(b)$, then there exists some $ x$ in $ [a,b]$ such that $ f(x) = c$.
	\end{quote}

	\begin{enumerate}
		\item Let $ A$ be the set of $ x$ in $ [a,b]$ such that $ f(x) \le c$. Why is the set $ A$ non-empty?

		\item 	As $ A$ is a non-empty subset of $ [a,b]$ by (P13) $ A$ has a least upper bound, let $ y = \sup A$. We want to show that $ f(y) = c$. We'll prove this by contradiction, assume on the contrary that $ f(y) \neq c$.
		      \begin{description}
		      	\item[Case I $ f(y) < c$ ] In this case argue that for some $ \delta$ the number $ y + \delta$ is in $ A$ and hence $ y$ cannot be an upper bound.
		      	\item[Case II $ f(y) > c$ ] In this case argue that for some $ \delta$ the number $ y - \delta$ is also an upper bound of $ A$ and hence $ y$ cannot be the \emph{least} upper bound.
		      \end{description}

		\item Give an example to show that the Intermediate Value Theorem does not hold if the function $ f$ is discontinuous. Also describe the set $ A$, defined as above, for your example and explain why the above proof fails for this case.
	\end{enumerate}


	% \item Proofs are not required for this problem, however describe your answers as precisely as possible.
	%
	% \begin{enumerate}
	% 	\item Find a function which is discontinuous at $1, \frac{1}{2}, \frac{1}{3},\frac{1}{4}, \ldots$ but continuous at all other points.
	% 	\item Find a function which is discontinuous at $1, \frac{1}{2}, \frac{1}{3},\frac{1}{4}, \ldots$ and 0 but continuous at all other points.
	% \end{enumerate}
	% If $ \lim \limits _ {x \rightarrow a} f(x)$ exists but is $ \neq f(a)$ then $ f$ is said to have a \textbf{removable singularity} at $ a$.
	% \begin{enumerate}[resume]
	% 	\item Give an example of a function $ f$ with a removable discontinuity at 0.
	% 	\item Given an example of a function $ f$ with a discontinuity at 0	which is not removable.
	% \end{enumerate}

\end{questions}











\end{document}
