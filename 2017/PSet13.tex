\documentclass[9pt, a4paper, oneside, reqno]{amsart}

\usepackage[final]{pdfpages}
\usepackage{wrapfig}

\usepackage{enumitem}
\usepackage{parskip}
\usepackage{fancyhdr}
\usepackage{color}
\usepackage{multicol}
\renewcommand{\thefootnote}{\fnsymbol{footnote}}


\newcommand{\hint}[1]{\footnote{\raggedleft\rotatebox{180}{Hint: #1\hfill}}}

\pagestyle{fancy}

\newlist{questions}{enumerate}{1}
\setlist[questions, 1]{label = \bf Q.\arabic*., itemsep=1em}

\lhead{\scshape Apurva Nakade}
\rhead{\scshape Honors Single Variable Calculus}
\renewcommand*{\thepage}{\small\arabic{page}}
\title{Problem Set 13}

\begin{document}

\maketitle
\thispagestyle{fancy}


\section*{Part 1 - Nowhere Differentiable Function}
In this problem set we'll construct a function $ f_\infty(x)$ which is continuous everywhere but differentiable nowhere(!) using only the techniques you've learnt in this course.


\begin{questions}
	\item
	Define the function $ \{ x \}$ as
	\begin{align*}
		\{x \} & = \mbox{the distance of $x$ from the nearest integer }
	\end{align*}
	Let $ n$ denote a positive integer. Define the function
	\begin{align*}
		f_n(x)
		 & =  \{x\} + \dfrac{\{ 2 x\}}{2} + \dots + \dfrac{\{ 2^n x\}}{2^{n}} \\
		 & = \sum \limits_{i=0}^{n} \dfrac{\{ 2^i x\}}{2^{i}}
	\end{align*}
 Draw the graph of the functions $ \{x\}$, $ \dfrac{\{ 2 x\}}{2}$, $ \dfrac{\{ 2^n x\}}{2^{n}}$, and $ f_1(x)$.


	\item Let $ x \in [0,1]$. Determine the relation between
	\begin{enumerate}
		\item $ \{x\}$ and $ \{x+1\}$
		\item $ \{x\}$ and $ \{1-x\}$
		\item $ \{x\}$ and $ \{x+1/2 \}$
		\item $ \{x\}$ and $ \{2x\}$
	\end{enumerate}

	\item Using properties of continuous functions argue that $ \{ x \}$ is a continuous function, and hence so is $ f_n(x)$.

	\item What are the points at which $ f_n(x)$ is not differentiable? Assuming that it makes sense to take limits of functions, what do you think are the points at which the function $ \lim \limits_{n\rightarrow \infty}f_n(x)$ is not differentiable. (Proof not needed.)

	\item Using your favorite test for series convergence, show that for every real number $ x$ the series $\sum \limits_{i=0}^{\infty} \dfrac{\{ 2^i x\}}{2^{i}}$ converges.
\end{questions}

Hence we can \textbf{define} a new function
\begin{align*}
	f_\infty(x) & = \sum \limits_{i=0}^{\infty} \dfrac{\{ 2^i x\}}{2^{i}}
\end{align*}
\begin{questions}[resume]
	\item Show that $ |f_\infty(x) - f_n(x)| \le 2^{-n}$ for all $ x$.

	\item Using the triangle inequality
	\begin{align*}
		|f_\infty(x) - f_\infty(a)|
		 & \le |f_\infty(x) - f_n(x)|              \\
		 & \qquad + |f_n(x) - f_n(a)|              \\
		 & \qquad  \qquad + |f_n(a) - f_\infty(a)|
	\end{align*}
	and the $ \epsilon-\delta$ definition of continuity, prove that $ f_\infty(x)$ is a continuous function.
\end{questions}
In order to show that $ f_\infty(x)$ is not differentiable anywhere it's helpful to use binary expansions of numbers.

Let $ x$ be a real number with binary expansion $ x = m + 0.a_1 a_2 a_3 \dots$ where $ m$ is an integer and each $ a_i = 0$ or $ 1$.
\begin{questions}[resume]
	\item Show that the number whose binary expansion consists of all one's $ 0.111 \dots 1 \dots $ is equal to 1.\hint{This is a geometric series.} What is the corresponding statement for decimal expansions?

	Because of this we can assume that there are no \emph{trailing 1's} in the binary expansion of any number.

	\item
	\begin{enumerate}
		\item Find $ \{ x \}$ in terms of the binary expansion of $ x$.
		\item Find $ \dfrac{\{ 2^i x\}}{2^{i}}$ in terms of the binary expansion of $ x$.
		\item If $ x$ has a finite binary expansion $ x = m + 0.a_1 a_2 \cdots a_n$, what is $ f_\infty(x)$?
	\end{enumerate}
\end{questions}

Let $ b_n$ be the position of the $ n^{th}$ zero after the decimal point in the binary expansion of $ x$.
	For example,
	\begin{quote}
		{if $ x = 1011.1010101\dots$ then $ b_n =2n$,}\\
	{if $ x = 0.1011011011\dots$ then $ b_n =3n-1$, etc.}
\end{quote}
\begin{questions}[resume]
	\item
	\begin{enumerate}
		\item Show that
		\begin{align*}
			\dfrac{\{ 2^i(x + 2^{-b_n})\}}{2^{i}} - \dfrac{\{ 2^i x\}}{2^{i}}
			&=
			\begin{cases}
				-2^{-b_n} & \mbox { if }i < b_n - 1 \\
				0 & \mbox { if }i \ge b_n
			\end{cases}
		\end{align*}
		\item Show that $ \dfrac{f_\infty(x + 2^{-b_n}) - f_\infty(x)}{2^{-b_n}} < -(b_n - 1) + 2^{-b_n+1}$.
	\end{enumerate}
	\item Show that $ \lim \limits_{n \rightarrow \infty} \dfrac{f_\infty(x + 2^{-b_n}) - f_\infty(x)}{2^{-b_n}}$ does not exist. Conclude that $ f(x)$ is not differentiable at $ x$.
\end{questions}





\newpage
\section*{Part 2 - Trigonometry and Complex Numbers}{
	Complex numbers are numbers of the form $ a+i.b$ where $ a,b$ are real numbers and $ i^2 = -1$. Complex numbers are very useful in calculus, especially for finding integrals, because of the following \textbf{Euler's identity}
	\begin{align*}
		e^{i \theta} = \cos \theta + i \sin \theta
	\end{align*}
}
\begin{questions}[resume]
	\item Verify Euler's identity using the Taylor series for $ e^x$, $ \sin x$ and $ \cos x$.
	\item Using the fact that $ e^{i(a+b)} = e^{ia}.e^{ib}$ compute the formulae for  $ \sin(a+b)$, $ \cos(a+b)$, $ \sin 2x$, $ \cos 2x$, $ \sin 3x$, and $ \cos 3x$.
	\item Show that
	\begin{align*}
		\cos x = \dfrac{e^{ix} + e^{-ix}}{2} &  & \sin x = \dfrac{e^{ix} - e^{-ix}}{2i}
	\end{align*}
	(Compare these with the formulae for $ \sinh x$ and $ \cosh x$.)
	\item Use these formulae to compute the following integrals\hint{You might need to use the identity $ \dfrac{1}{a+ib} = \dfrac{a-ib}{a^2 + b^2}$}
	\begin{enumerate}
		\item $ e^{ax} \sin bx$
		\item $ e^{ax} \cos bx$
		\item $ \cos^2 x$
		\item $ \sin x \cos 4x$
	\end{enumerate}
\end{questions}


\end{document}
