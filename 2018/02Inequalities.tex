%!TEX root=ClassNotes.tex

\section{Inequalities}
In this section we'll learn how to solve inequalities and in the process figure out how to write proofs of statements involving inequalities.


The biggest different between equalities and inequalities is that, unlike equalities, there are usually infinitely many (if any) solutions to inequalities. When we we're trying to solve an inequality we're not looking for the best solution, we're not even looking for a good solution, we're simply looking for {\bf one} solution. For example, both $x = 1.001$ and $x = 10^{100}$ are equally valid solutions to the inequality $x > 1$.

\begin{exercise}
	Find a positive real solution to each of the following inequalities.
	\begin{enumerate}
		\item $x ^ 3 + x < 1$
		\item $(1 + x)^2 - 1 < 1$
		\item $1 - (1 - x)^2 < 1$
		\item $x ^ 3 + 2x < 1 + x^2$
		\item $x^{10} + x > 10 $
		\item $x^3 - x > 1$
	\end{enumerate}
\end{exercise}

When we try to solve equalities we try to simply the equation until it becomes easy to solve. The same thing is true for inequalities, however, the ways to simplify an inequality are much more subtle.

\begin{example}
	If we're trying to solve $P(x) > \epsilon$, where $P$ is some complicated expression that cannot be simplified, then we try to find some simpler $Q(x)$, such that $P(x) > Q(x)$ and try to solve for $Q(x) > \epsilon$ instead. Similarly, if we're trying to solve $P(x) < \epsilon$, then we try to find some simpler $Q(x)$, such that $P(x) < Q(x)$ and try to solve for $Q(x) < \epsilon$ instead.

	 There is no {\it canonical way} in which an inequality can be solved. We somehow simplify the inequality and hope that we get a solution.
\end{example}

For solving complicated inequalities, we need a systematic approach towards estimating functions. There are different tricks and identities for estimating different kinds of functions. For now, we'll focus mainly on estimating polynomials.

\begin{exercise}$ $
	\label{q:inequality_1}
	Prove the following extremely useful set of inequalities
	\begin{center}
		\begin{tabular}{ll}
			if $1 \ge x > 0$ &then $ 1 \ge x \ge x^2 \ge x^3 \ge \dots$ \\
			if $1 \le x$ &then $1 \le x \le x^2 \le x^3 \le \dots$.
		\end{tabular}
	\end{center}
\end{exercise}


\noindent {\bf Word of Caution: } You should be very careful when negative numbers are involved. Multiplying an inequality by a negative number changes it's sign, for example, $2 < 3$ but $-2 > -3$. So,
\begin{center}
	\begin{tabular}{ll}
		if $1 \ge x > 0$ &then $ -1 \le -x \le -x^2 \le -x^3 \le \dots$ \\
		if $1 \le x$ &then $-1 \ge -x \ge -x^2 \ge -x^3 \ge \dots$.
	\end{tabular}
\end{center}


\subsection{Proofs involving Inequalities}
We'll encounter several statements in this class like: {\it For every $\epsilon> 0$ there exists a positive real number $x$ such that $x^3 + 3x < \epsilon$.} When we want to prove such a statement, we are essentially trying to solve $x^3 + 3x < \epsilon$. (Why?) For example,\\
\begin{q}
	\label{q:sample_problem_inequality_1}
	Prove that for every $\epsilon > 0$ there exists a positive real number $x$ such that $x^3 + 3x < \epsilon$.
\end{q}
\noindent We're asking for a solution to the equation $x^3 + 3x < \epsilon$ for an arbitrary positive real number $\epsilon$.
\begin{enumerate}
	\item The terms $x^3$ and $x$ have different degrees. We can try to use the inequalities in Exercise \ref{q:inequality_1} to simplify their sum, for this we need to know if $x \ge 1$ or $x \le 1$.
	\item We're probably looking for a small  $x$ so we could assume $x \le 1$. If this does not work we'll come back try something else. (Remember that we're trying to find {\bf one} solution.)
	\item As $x \le 1$, $x^3 \le x$ so that
		$x^3 + 3x \le x + 3x= 4x$.
	\item	As $x^3 + 3x \le 4x$ and we want to find a solution to $x^3 + 3x < \epsilon$, it suffices to solve $4x < \epsilon$. As $\epsilon$ is positive, \textbf{one} solution for this is $x = \epsilon / 5$.
	\item We still need to check if this solution actually works. We can do this by trying to write down a direct proof and making sure that all the implications are logically sound.
	\item We required the condition $x \le 1$ for this solution to work. To ensure this we'll set $x = \min \{ 1, {\epsilon}/{5}\}$.
\end{enumerate}

Now we know what the solution is, but we still need to write down a proof for the original statement. A proof should always start with some assumptions and logically derive the required conclusion. For us the assumption will be $\epsilon > 0$ and $x = \min \{ 1, {\epsilon}/{5}\}$ and the expected conclusion is $x^3 + 3x < \epsilon$.

\begin{proof}[Proof of Q. \ref{q:sample_problem_inequality_1}]
	Let $\epsilon > 0$ and let $x = \min \{ 1, {\epsilon}/{5}\}$. Note that $x \le 1$ and hence $x^3 \le x$. Then,
	\begin{align*}
		x^3 + 3x
		&\le x + 3x \\
		&= 4x \\
		&= 4 \min  \{ 1, {\epsilon}/{5}\} \\
		&\le 4 \epsilon /5 \\
		&< \epsilon & \mbox{ as $\epsilon$ is positive}
	\end{align*}
	Hence, $x= \min \{ 1, {\epsilon}/{5}\}$ is a solution of $x^3 + 3x < \epsilon$, which proves the proposition.
\end{proof}
First we had to solve the inequality then reverse the steps: start with the solution and write a {\it direct proof} for the proposition. This is how proofs are usually discovered. You make an ``educated guess'' and hope that it works. Sometimes it doesn't, so you go back to finding another ``educated guess''.


\begin{remark}[A note on quantifiers]
	 For proving the statement {\it``prove that for every $\epsilon > 0$ there exists a positive real number $x$ such that $x^3 + 3x < \epsilon$''} we had to start with the statement {\it ``let $\epsilon > 0$ and $x = \min \{ 1, {\epsilon}/{5}\}$''}.
	 The variable $x$ depends on the variable $\epsilon$.
	 This is fine because of the order of quantifiers: {\it``\dots for every $\epsilon > 0$ there exists a positive real number $x$ \dots''}; the quantifier for $\epsilon$ comes before the quantifier for $x$ and hence the variable $x$ can depend on the variable $\epsilon$.

	If instead, suppose we were trying to prove {\it ``there exists a positive real number $x$ such that for every $\epsilon > 0$, $x^3 + 3x < \epsilon$''}.
	We're still trying to solve the equation $x^3 + 3x < \epsilon$ but now the quantifier for $x$ comes before the quantifier for $\epsilon$ and hence $x$ cannot depend on the variable $\epsilon$ but instead should be a constant that universally solves the equation $x^3 + 3x < \epsilon$ for every $\epsilon > 0$.
	No such $x$ exists (why?) and hence the statement {\it ``there exists a positive real number $x$ such that for every $\epsilon > 0$, $x^3 + 3x < \epsilon$''} is false.
\end{remark}

\begin{exercise}
	Prove that for every $\epsilon > 0$ there exists a positive real number $x$ such that \dots
		\begin{enumerate}
			\item  $x ^ 3 + x < \epsilon$.
			\item  $(1 + x)^2 - 1 < \epsilon$.
			\item $1 - (1 - x)^2 < \epsilon$.
			\item  $x ^ 3 + 2x < \epsilon + x^2$.
			\item  $x^{10} + x > \epsilon $.
			\item  $x^3 - x > \epsilon $.
						\hint{Break $x^3$ as $\frac{x^3}{2} + \frac{x^3}{2}$ and find the conditions on $x$ for which $\left(\frac{x^3}{2} - x\right) > 0$.}\\
		\end{enumerate}
\end{exercise}

Finally, we are sometimes required to find not one solution but a range of solutions. We try to come up an ``educated guess'' by the same method, and many a times this gives us the required range for free with a few changes in the final proof.

\begin{q}
	\label{q:sample_problem_inequality_2}
	Prove that for every $\epsilon > 0$, there exists a positive real number $\delta$ such that, for all $x$, if $0 < x < \delta$ then $x^3 + 3x < \epsilon$.
\end{q}
\begin{proof}
	Let $\epsilon > 0$, let $\delta = \min \{ 1, \epsilon/5\}$ and let $ 0 < x < \delta$. Because $\delta \le 1$, we have $x < 1$ and hence $x^3 < x$. As before,
	\begin{align*}
		x^3 + 3x
		&\le x + 3x \\
		&= 4x \\
		&< 4 \delta \\
		&= 4 \min  \{ 1, {\epsilon}/{5}\} \\
		&\le 4 \epsilon /5 \\
		&< \epsilon& \mbox{ as $\epsilon$ is positive}
	\end{align*}
	Hence, every $0 < x <  \min \{ 1, \epsilon/5\}$ is a solution of $x^3 + 3x < \epsilon$, which proves the proposition.
\end{proof}


\begin{exercise}
	Prove that for every $\epsilon > 0$, there exists a positive real number $\delta$ such that, for all $x$, if $ 0 < x < \delta$ then \dots
		\begin{enumerate}
			\item  $x ^ 3 + x < \epsilon$.
			\item  $(1 + x)^2 - 1 < \epsilon$.
			\item $1 - (1 - x)^2 < \epsilon$.
			\item  $x ^ 3 + 2x < \epsilon + x^2$.
		\end{enumerate}
		\noindent Prove that for every $\epsilon > 0$, there exists a positive real number $\delta$ such that, for all $x$, if $ x > \delta$ then
		\begin{enumerate}
			\item $x^{10} + x > \epsilon $.
			\item $x^3 - x > \epsilon$.
		\end{enumerate}
\end{exercise}

\subsubsection*{Optional Problems}
\begin{exercise}
	\begin{enumerate}
		\item Prove that for every $\epsilon > 0$, for every real number $x > 0$, there exists a $\delta > 0$ such that, for all real numbers $y > 0$, if $ 0 < y - x < \delta$ then $y^2 - x^2 < \epsilon$.
		\item Prove that the following statement is false: For every $\epsilon > 0$, there exists a $\delta > 0$ such that, for every real number $x > 0$, for all real numbers $y > 0$, if $ 0 < y - x < \delta$ then $y^2 - x^2 < \epsilon$.
	\end{enumerate}
\end{exercise}
